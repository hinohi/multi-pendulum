\documentclass[uplatex]{jsarticle}

\usepackage[japanese]{babel}

% Useful packages
\usepackage{amsmath}
\usepackage{graphicx}
\usepackage[colorlinks=true, allcolors=blue]{hyperref}
\usepackage{physics}
\usepackage{url}
\usepackage{siunitx}

\title{多重振り子}
\author{中山 大樹}

\begin{document}
\maketitle

\section{2次元単振り子をラグランジュ未定乗数法で解く}

質量$m$[\si{kg}]の質点が伸び縮みしない剛体棒で繋がれている状況を考える。
重力の向きを$-y$方向にとり、重力加速度を$g$とする。

振り子の根本の座標を$(x_0(t), y_0(t))$、質点の座標を$(x_1(t), y_1(t))$とする。

拘束条件は
\begin{align}
    \qty(x_1 - x_0)^2 + \qty(y_1 - y_0)^2 = l^2  \label{2d_single_cons}
\end{align}
ただし$l$[\si{m}]は剛体棒の長さ。

拘束条件込みのラグランジアンは\cite{a}を参考にすると
\begin{align}
    L = \frac{1}{2} m \qty(\dot{x}_1^2 + \dot{y}_1^2) - mgy_1
        - \lambda \qty(\qty(x_1 - x_0)^2 + \qty(y_1 - y_0)^2 - l^2)
\end{align}
となる。
オイラーラグランジュの運動方程式を
\begin{align}
    \dv{}{t}\pdv{L}{\dot{q}} - \pdv{L}{q} = 0
\end{align}
より求めておくと
\begin{align}
\begin{cases}
    m \ddot{x}_1 = -2\lambda\qty(x_1 - x_0) \\
    m \ddot{y}_1 = -2\lambda\qty(y_1 - y_0) - mg
\end{cases}
\label{2d_single_eom}
\end{align}
となる。

$x_0(t)$と$y_0(t)$は外から与えられる2階時間微分可能な関数である。

\eqref{2d_single_cons}を時間微分する
\begin{align}
    &\qty(x_1 - x_0)^2 + \qty(y_1 - y_0)^2 = l^2 \tag{\ref{2d_single_cons}} \\
    \xrightarrow{\dv{}{t}}&
    \qty(x_1 - x_0) \qty(\dot{x}_1 - \dot{x}_0) + \qty(y_1 - y_0) \qty(\dot{y}_1 - \dot{y}_0) = 0 \\
    \xrightarrow{\dv{}{t}}&
    \qty(\dot{x}_1 - \dot{x}_0)^2 + \qty(\dot{y}_1 - \dot{y}_0)^2 
        + \qty(x_1 - x_0) \qty(\ddot{x}_1 - \ddot{x}_0) + \qty(y_1 - y_0) \qty(\ddot{y}_1 - \ddot{y}_0)= 0
        \label{2d_single_cons_dd}
\end{align}
この\eqref{2d_single_cons_dd}に\eqref{2d_single_eom}を代入して$\lambda(t)$について解くと
\begin{align}
    0 &= \qty(\dot{x}_1 - \dot{x}_0)^2 + \qty(\dot{y}_1 - \dot{y}_0)^2
        + \qty(x_1 - x_0)\qty(-2\frac{\lambda}{m}\qty(x_1 - x_0) - \ddot{x}_0)
        + \qty(y_1 - y_0)\qty(-2\frac{\lambda}{m}\qty(y_1 - y_0) - g - \ddot{y}_0) \nonumber \\
    &= \qty(\dot{x}_1 - \dot{x}_0)^2 + \qty(\dot{y}_1 - \dot{y}_0)^2
        -2 l^2 \frac{\lambda}{m}
        - \qty(x_1 - x_0) \ddot{x}_0
        - \qty(y_1 - y_0)\qty(g + \ddot{y}_0) \nonumber \\
    \rightarrow&
    \frac{\lambda}{m} = \frac{1}{2 l^2} \qty(
        \qty(\dot{x}_1 - \dot{x}_0)^2 + \qty(\dot{y}_1 - \dot{y}_0)^2
        - \qty(x_1 - x_0) \ddot{x}_0 - \qty(y_1 - y_0)\qty(g + \ddot{y}_0)
    ) \label{2d_single_lambda}
\end{align}
途中で剛体棒長さの拘束条件\eqref{2d_single_cons}がそのまま出てくるので$l^2$に置き換えられるのがミソですね。
あとはこの\eqref{2d_single_lambda}を\eqref{2d_single_eom}に代入して普通に運動方程式を解けばいいはず。

一般の場合を考える前に特殊な場合を考えておく。
\eqref{2d_single_lambda}で$x_0(t)=y_0(t)=0$の場合を考える、つまりは普通のなんの変哲もない振り子ということ。
\eqref{2d_single_lambda}より
\begin{align}
    \frac{\lambda}{m} = \frac{1}{2 l^2} \qty(\dot{x}_1^2 + \dot{y}_1^2 - y_1 g)
\end{align}
\eqref{2d_single_eom}に代入して
\begin{align}
\begin{cases}
    \ddot{x}_1 = -\frac{1}{l^2} \qty(\dot{x}_1^2 + \dot{y}_1^2 - y_1 g) x_1 \\
    \ddot{y}_1 = -\frac{1}{l^2} \qty(\dot{x}_1^2 + \dot{y}_1^2 - y_1 g) y_1 - g
\end{cases}
\end{align}
となる。
あとでこれを直接解いてみるが、一旦検算のために$x_1(t)=l\cos{\theta(t)}, y_1(t)=l\sin{\theta(t)}$を代入してみる
(これは$x$軸から反時計回りに$\theta$をとっている、つまり普通の偏角)。

$\ddot{x}_1$についての式より
\begin{align}
    0 &= l\dv[2]{\qty(\cos\theta(t))}{t}
        + \frac{1}{l^2} \qty(
            \qty(l\dv{\qty(\cos\theta(t))}{t})^2 + \qty(l\dv{\qty(\sin\theta(t))}{t})^2
            - l\sin{\theta(t)} g
        ) l\cos{\theta(t)} \nonumber \\
    &= -\dv{}{t}\qty(\dot{\theta}\sin\theta) + \qty(\dot{\theta}^2 - \frac{g}{l}\sin\theta) \cos\theta \nonumber \\
    &= -\ddot{\theta}\sin\theta - \dot{\theta}^2\cos\theta + \qty(\dot{\theta}^2 - \frac{g}{l}\sin\theta) \cos\theta \nonumber \\
    \rightarrow&
    \ddot{\theta} = -\frac{g}{l}\cos\theta
\end{align}
$\ddot{y}_1$についての式も同様に変形できる。
あってそうですね。

TODO: ここに数値計算

\section{3次元多重振り子をラグランジュ未定乗数法で解く}

TODO

\bibliographystyle{plain}
\bibliography{main.bib}
\end{document}
