\documentclass[uplatex]{jsarticle}

\usepackage[japanese]{babel}

% Useful packages
\usepackage{amsmath}
\usepackage{graphicx}
\usepackage[colorlinks=true, allcolors=blue]{hyperref}
\usepackage{physics}
\usepackage{url}
\usepackage{siunitx}

\title{多重振り子}
\author{中山 大樹}

\begin{document}
\maketitle

\section{2次元単振り子をラグランジュ未定乗数法で解く}

質量$m$[\si{kg}]の質点が伸び縮みしない剛体棒で繋がれている状況を考える。
重力の向きを$-y$方向にとり、重力加速度を$g$とする。

振り子の根本の座標を$(x_0(t), y_0(t))$、質点の座標を$(x_1(t), y_1(t))$とする。

拘束条件は
\begin{align}
    \qty(x_1 - x_0)^2 + \qty(y_1 - y_0)^2 = l^2  \label{2d_single_cons}
\end{align}
ただし$l$[\si{m}]は剛体棒の長さ。

拘束条件込みのラグランジアンは\cite{a}を参考にすると
\begin{align}
    L = \frac{1}{2} m \qty(\dot{x}_1^2 + \dot{y}_1^2) - mgy_1
        - \lambda \qty(\qty(x_1 - x_0)^2 + \qty(y_1 - y_0)^2 - l^2)
\end{align}
となる。
オイラーラグランジュの運動方程式を
\begin{align}
    \dv{}{t}\pdv{L}{\dot{q}} - \pdv{L}{q} = 0
\end{align}
より求めておくと
\begin{align}
\begin{cases}
    m \ddot{x}_1 = -2\lambda\qty(x_1 - x_0) \\
    m \ddot{y}_1 = -2\lambda\qty(y_1 - y_0) - mg
\end{cases}
\label{2d_single_eom}
\end{align}
となる。

$x_0(t)$と$y_0(t)$は外から与えられる2階時間微分可能な関数である。

\eqref{2d_single_cons}を時間微分する
\begin{align}
    &\qty(x_1 - x_0)^2 + \qty(y_1 - y_0)^2 = l^2 \tag{\ref{2d_single_cons}} \\
    \xrightarrow{\dv{}{t}}&
    \qty(x_1 - x_0) \qty(\dot{x}_1 - \dot{x}_0) + \qty(y_1 - y_0) \qty(\dot{y}_1 - \dot{y}_0) = 0 \\
    \xrightarrow{\dv{}{t}}&
    \qty(\dot{x}_1 - \dot{x}_0)^2 + \qty(\dot{y}_1 - \dot{y}_0)^2 
        + \qty(x_1 - x_0) \qty(\ddot{x}_1 - \ddot{x}_0) + \qty(y_1 - y_0) \qty(\ddot{y}_1 - \ddot{y}_0)= 0
        \label{2d_single_cons_dd}
\end{align}
この\eqref{2d_single_cons_dd}に\eqref{2d_single_eom}を代入して$\lambda(t)/m$について解くと
\begin{align}
    0 &= \qty(\dot{x}_1 - \dot{x}_0)^2 + \qty(\dot{y}_1 - \dot{y}_0)^2
        + \qty(x_1 - x_0)\qty(-2\frac{\lambda}{m}\qty(x_1 - x_0) - \ddot{x}_0)
        + \qty(y_1 - y_0)\qty(-2\frac{\lambda}{m}\qty(y_1 - y_0) - g - \ddot{y}_0) \nonumber \\
    &= \qty(\dot{x}_1 - \dot{x}_0)^2 + \qty(\dot{y}_1 - \dot{y}_0)^2
        -2 l^2 \frac{\lambda}{m}
        - \qty(x_1 - x_0) \ddot{x}_0
        - \qty(y_1 - y_0)\qty(g + \ddot{y}_0) \nonumber \\
    \rightarrow&
    \frac{\lambda}{m} = \frac{1}{2 l^2} \qty(
        \qty(\dot{x}_1 - \dot{x}_0)^2 + \qty(\dot{y}_1 - \dot{y}_0)^2
        - \qty(x_1 - x_0) \ddot{x}_0 - \qty(y_1 - y_0)\qty(g + \ddot{y}_0)
    ) \label{2d_single_lambda}
\end{align}
途中で剛体棒長さの拘束条件\eqref{2d_single_cons}がそのまま出てくるので$l^2$に置き換えられるのがミソですね。
あとはこの\eqref{2d_single_lambda}を\eqref{2d_single_eom}に代入して普通に運動方程式を解けばいいはず。

一般の場合を考える前に特殊な場合を考えておく。
\eqref{2d_single_lambda}で$x_0(t)=y_0(t)=0$の場合を考える、つまりは普通のなんの変哲もない振り子ということ。
\eqref{2d_single_lambda}より
\begin{align}
    \frac{\lambda}{m} = \frac{1}{2 l^2} \qty(\dot{x}_1^2 + \dot{y}_1^2 - y_1 g)
\end{align}
\eqref{2d_single_eom}に代入して
\begin{align}
\begin{cases}
    \ddot{x}_1 = -\frac{1}{l^2} \qty(\dot{x}_1^2 + \dot{y}_1^2 - y_1 g) x_1 \\
    \ddot{y}_1 = -\frac{1}{l^2} \qty(\dot{x}_1^2 + \dot{y}_1^2 - y_1 g) y_1 - g
\end{cases}
\end{align}
となる。
あとでこれを直接解いてみるが、一旦検算のために$x_1(t)=l\cos{\theta(t)}, y_1(t)=l\sin{\theta(t)}$を代入してみる
(これは$x$軸から反時計回りに$\theta$をとっている、つまり普通の偏角)。

$\ddot{x}_1$についての式より
\begin{align}
    0 &= l\dv[2]{\qty(\cos\theta(t))}{t}
        + \frac{1}{l^2} \qty(
            \qty(l\dv{\qty(\cos\theta(t))}{t})^2 + \qty(l\dv{\qty(\sin\theta(t))}{t})^2
            - l\sin{\theta(t)} g
        ) l\cos{\theta(t)} \nonumber \\
    &= -\dv{}{t}\qty(\dot{\theta}\sin\theta) + \qty(\dot{\theta}^2 - \frac{g}{l}\sin\theta) \cos\theta \nonumber \\
    &= -\ddot{\theta}\sin\theta - \dot{\theta}^2\cos\theta + \qty(\dot{\theta}^2 - \frac{g}{l}\sin\theta) \cos\theta \nonumber \\
    \rightarrow&
    \ddot{\theta} = -\frac{g}{l}\cos\theta
\end{align}
$\ddot{y}_1$についての式も同様に変形できる。
あってそうですね。

\section{多次元多重振り子をラグランジュ未定乗数法で解く}

$N$個の質点が伸び縮みしない剛体棒で繋がれている状況を考える。
根本の座標が$\va{x}_0(t)$、根本から$i$番目の質点の質量$m_i$座標$\va{x}_i(t)$とする。
$0$番目が根本である。

今回は問題設定として、任意の$i$で座標を given にできるとする。
つまり多重振り子の任意の箇所を手で持って操作できるような状況を考える。
操作対象の点(根本の点を含むので質点とは言わない)のインデックスの集合を$C$としておく。

重力加速度を$\va{g}$とする。
このベクトルになった重力加速度は、例えば1章の2次元振り子の例だと$\va{g} = (0, g)$と解釈すれば良い。
重力の向きと重力加速度の向きがマイナス符号になってしまっているな。

これから何度も出てくるので以下の略記を利用する。
\begin{align}
    \va{x}_{i-1,i} = \va{x}_i - \va{x}_{i-1}
\end{align}

剛体棒による拘束条件を定式化すると
\begin{align}
    \abs{\va{x}_{i-1,i}}^2 = l_i^2 \label{nd_mult_cons}
\end{align}
となる。

拘束条件込みのラグランジアンと運動方程式は
\begin{align}
    L &= \sum_{i=1}^N \qty(
        \frac{1}{2} m_i \abs{\dot{\va{x}}_i}^2
        - m_i \va{g} \vdot \va{x}_i
        - \lambda_i \qty(\abs{\va{x}_{i-1,i}}^2 - l_i^2)
    ) \\
    & \begin{cases}
        m_i \ddot{\va{x}}_i = -m_i \va{g} - 2\lambda_i \va{x}_{i-1,i} + 2\lambda_{i+1} \va{x}_{i,i+1} & i < N \\
        m_i \ddot{\va{x}}_i = -m_i \va{g} - 2\lambda_i \va{x}_{i-1,i} & i = N \\
    \end{cases}
    \label{ng_multi_eom}
\end{align}
となる。

無次元化しておく。
適当な単位長さ$l$、単位質量$m$を定義し、典型時間$\tau$も素朴に定義する。
\begin{align}
    \tau &= \sqrt{\frac{l}{\abs{\va{g}}}}
\end{align}
これを用いて以下の書き換えを行う。
\begin{align}
    x_i^\prime = x_i / l \\
    t^\prime = t / \tau \\
    \va{g}^\prime = \tau^2 \va{g} / l \\
    \lambda_i^\prime = \lambda_i \tau^2 / m \\
    m_i^\prime = m_i / m
\end{align}
まず運動方程式\eqref{ng_multi_eom}の$i<N$の側が
\begin{align}
    & m_i \ddot{\va{x}}_i = -m_i \va{g} - 2\lambda_i \va{x}_{i-1,i} + 2\lambda_{i+1} \va{x}_{i,i+1} \tag{\ref{ng_multi_eom}} \\
    \rightarrow &
    \frac{l}{\tau^2} \ddot{\va{x}}_i^\prime =
        -\frac{l}{\tau^2}\va{g}^\prime
        -2\frac{\lambda_i^\prime}{m_i^\prime \tau^2} l \va{x}_{i-1,i}^\prime
        +2\frac{\lambda_{i+1}^\prime}{m_i^\prime \tau^2} l \va{x}_{i,i+1}^\prime \nonumber \\
    \rightarrow &
    \ddot{\va{x}}_i^\prime =
        -\va{g}^\prime
        -2\frac{\lambda_i^\prime}{m_i^\prime} \va{x}_{i-1,i}^\prime
        +2\frac{\lambda_{i+1}^\prime}{m_i^\prime} \va{x}_{i,i+1}^\prime
\end{align}
となる。

\eqref{nd_mult_cons}を時間で微分する
\begin{align}
    & \abs{\va{x}_{i-1,i}}^2 = l_i^2 \tag{\ref{nd_mult_cons}} \\
    \xrightarrow{\dv[2]{}{t}} &
    \abs{\dot{\va{x}}_{i-1,i}}^2 + \va{x}_{i-1,i} \vdot \ddot{\va{x}}_{i-1,i} = 0 \\
    \xrightarrow{無次元化} &
    \abs{\dot{\va{x}}_{i-1,i}^\prime}^2 + \va{x}_{i-1,i}^\prime \vdot \ddot{\va{x}}_{i-1,i}^\prime = 0
\end{align}
これを使って$\lambda_i/m_i$について解く。

面倒なので以下は$0 \in C$、$i \notin C (i > 0)$かつ$N > 1$とする。
まずは$i=1$の場合
\begin{align}
    0 &= \abs{\dot{\va{x}}_{0,1}^\prime}^2 + \va{x}_{0,1}^\prime
        \vdot \qty(\ddot{\va{x}}_1^\prime - \ddot{\va{x}}_0^\prime) \nonumber \\
    &= \abs{\dot{\va{x}}_{0,1}^\prime}^2 + \va{x}_{0,1}^\prime
        \vdot \qty(
            -\va{g}^\prime - 2\frac{\lambda_1^\prime}{m_1^\prime} \va{x}_{0,1}^\prime
            + 2\frac{\lambda_2^\prime}{m_1^\prime}\va{x}_{1,2}^\prime - \ddot{\va{x}}_0^\prime) \nonumber \\
    \rightarrow &
    2\frac{\abs{\va{x}_{0,1}^\prime}^2}{m_1^\prime} \lambda_1^\prime
    -2\frac{\va{x}_{0,1}^\prime \vdot \va{x}_{1,2}^\prime}{m_1^\prime} \lambda_2^\prime
        = \abs{\dot{\va{x}}_{0,1}^\prime}^2 - \va{x}_{0,1}^\prime \vdot \qty(\va{g}^\prime + \ddot{\va{x}}_0^\prime)
\end{align}
$1<i<N$の場合
\begin{align}
    0 &= \abs{\dot{\va{x}}_{i-1,i}^\prime}^2 + \va{x}_{i-1,i}^\prime
        \vdot \qty(\ddot{\va{x}}_i^\prime - \ddot{\va{x}}_{i-1}^\prime) \nonumber \\
    &= \abs{\dot{\va{x}}_{i-1,i}^\prime}^2 + 2\va{x}_{i-1,i}^\prime \vdot \qty(
        -\frac{\lambda_i^\prime}{m_i^\prime} \va{x}_{i-1,i}^\prime
        +\frac{\lambda_{i+1}^\prime}{m_i^\prime} \va{x}_{i,i+1}^\prime
        +\frac{\lambda_{i-1}^\prime}{m_{i-1}^\prime} \va{x}_{i-2,i-1}^\prime
        -\frac{\lambda_i^\prime}{m_{i-1}^\prime} \va{x}_{i-1,i}^\prime
    ) \nonumber \\
    \rightarrow &
    -2\frac{\va{x}_{i-2,i-1}^\prime \vdot \va{x}_{i-1,i}^\prime}{m_{i-1}^\prime} \lambda_{i-1}^\prime
    +2\abs{\va{x}_{i-1,i}^\prime}^2 \qty(\frac{1}{m_i^\prime} + \frac{1}{m_{i-1}^\prime}) \lambda_i^\prime
    -2\frac{\va{x}_{i-1,i}^\prime \vdot \va{x}_{i,i+1}^\prime}{m_i^\prime} \lambda_{i+1}^\prime
    = \abs{\dot{\va{x}}_{i-1,i}^\prime}^2
\end{align}
$i=N$の場合
\begin{align}
    0 &= \abs{\dot{\va{x}}_{N-1,N}^\prime}^2 + \va{x}_{N-1,N}^\prime
        \vdot \qty(\ddot{\va{x}}_N^\prime - \ddot{\va{x}}_{N-1}^\prime) \nonumber \\
    &= \abs{\dot{\va{x}}_{N-1,N}^\prime}^2 + 2\va{x}_{N-1,N}^\prime \vdot \qty(
        -\frac{\lambda_N^\prime}{m_N^\prime} \va{x}_{N-1,N}^\prime
        +\frac{\lambda_{N-1}^\prime}{m_{N-1}^\prime} \va{x}_{N-2,N-1}^\prime
        -\frac{\lambda_N^\prime}{m_{N-1}^\prime} \va{x}_{N-1,N}^\prime
    ) \nonumber \\
    \rightarrow &
    -2\frac{\va{x}_{N-2,N-1}^\prime \vdot \va{x}_{N-1,N}^\prime}{m_{N-1}^\prime} \lambda_{N-1}^\prime
    +2\abs{\va{x}_{N-1,N}^\prime}^2 \qty(\frac{1}{m_N^\prime} + \frac{1}{m_{N-1}^\prime}) \lambda_N^\prime
    = \abs{\dot{\va{x}}_{N-1,N}^\prime}^2
\end{align}
これは
\begin{align}
    \begin{bmatrix}
        c   & b_1 &      0 & \dots   &         & 0 \\
        b_1 & a_2 &    b_2 &         &         & \\
        0   & b_2 & \ddots & \ddots  &         &  \\
        \vdots &  & \ddots & a_{N-2} & b_{N-2} & 0 \\
            &     &        & b_{N-2} & a_{N-1} & b_{N-1} \\
        0   &     &        &       0 & b_{N-1} & a_N
    \end{bmatrix}
    \begin{bmatrix}
        \lambda_1 \\ \lambda_2 \\ \vdots \\ \lambda_i \\ \vdots \\ \lambda_N
    \end{bmatrix}
    =
    \begin{bmatrix}
        \abs{\dot{\va{x}}_{0,1}^\prime}^2 - \va{x}_{0,1}^\prime \vdot \qty(\va{g}^\prime +\ddot{\va{x}}_0^\prime) \\
        \abs{\dot{\va{x}}_{1,2}^\prime}^2 \\
        \vdots \\
        \abs{\dot{\va{x}}_{i-1,i}^\prime}^2 \\
        \vdots \\
        \abs{\dot{\va{x}}_{N-1,N}^\prime}^2 \\
    \end{bmatrix}
\end{align}
ただし
\begin{align}
    c &= 2\frac{\abs{\va{x}_{0,1}^\prime}^2}{m_1^\prime} \\
    a_i &= 2\abs{\va{x}_{i-1,i}^\prime}^2 \qty(\frac{1}{m_i^\prime} + \frac{1}{m_{i-1}^\prime}) \\
    b_i &= -2\frac{\va{x}_{i-1,i}^\prime \vdot \va{x}_{i,i+1}^\prime}{m_i^\prime}
\end{align}


\bibliographystyle{plain}
\bibliography{main.bib}
\end{document}
